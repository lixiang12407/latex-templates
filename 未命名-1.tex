\documentclass[UTF8]{ctexart}

\usepackage{geometry}
\geometry{left=2cm,right=2cm,top=2.5cm,bottom=2cm}

\title{你好,world!}
\author{Liam}
\date{\today}

%---导言区---
\usepackage{fancyhdr} %调用宏包

% ---基本设置---

%设定页面的页眉页脚类型,$\LaTeX$内置了四种:empty、plain、headings及myheadings,但是我们现在不用这些内置的样式。
\pagestyle{fancy}

%清除原页眉页脚样式
\fancyhf{} 

%R:页面右边;O:奇数页;\leftmark:表示“一级标题”
\fancyhead[RO]{\leftmark}

%L:页面左边;E:偶数页;\rightmark:表示“二级标题”
\fancyhead[LE]{\rightmark}

%C:页面中间
\fancyhead[CO, CE]{文章题目}

%同上,但是不同位置放置不同信息
\fancyhead[LO, RE]{其他信息}

% 设置页脚,页眉的位置上也可以放置页码
\fancyfoot[RO, LE]{80301920}
\fancyfoot[LO, RE]{其他信息}

% 设置页眉页脚横线及样式
%页眉线宽,设为0可以去页眉线
\renewcommand{\headrulewidth}{0.3mm} 
%页脚线宽,设为0可以去页眉线
\renewcommand{\footrulewidth}{0.1mm} 

\begin{document}

\tableofcontents
\section{你好中国}
中国在East Asia.
\subsection{Hello Beijing}
北京是capital of China.
\subsubsection{Hello Dongcheng District}
\paragraph{Tian'anmen Square}
is in the center of Beijing
\subparagraph{Chairman Mao}
is in the center of 天安门广场。
\subsection{Hello 山东}
\paragraph{山东大学} is one of the best university in 山东。
\end{document}