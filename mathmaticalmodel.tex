\documentclass[12pt, a4paper, oneside]{ctexart}
\usepackage{amsmath, amsthm, amssymb, appendix, bm, fancyhdr, graphicx, geometry, mathrsfs, zhnumber}
\usepackage[bookmarks=true, colorlinks, citecolor=black, linkcolor=black]{hyperref}
\usepackage[framed, numbered, autolinebreaks, useliterate]{mcode}

\linespread{1.2}
\geometry{left=2.5cm, right=2.5cm, top=2.5cm, bottom=2.5cm}

\renewcommand{\thesection}{\zhnum{section}}
\renewcommand{\thesubsection}{\arabic{section}.\arabic{subsection}}
\renewcommand{\theequation}{\arabic{section}.\arabic{equation}}
\renewcommand{\thetable}{\arabic{section}.\arabic{table}}
\renewcommand{\thefigure}{\arabic{section}.\arabic{figure}}
\graphicspath{{./graphics/}}

\begin{document}

\pagestyle{empty}
\setcounter{page}{0}

\begin{center}
    \Large{\textbf{\LaTeX 建模论文模板3.0}}
\end{center}

\begin{center}
    \Large{\textbf{摘要}}
\end{center}

这里是摘要。

\textbf{关键词:}这里是关键词;这里是关键词。

\newpage
\setcounter{page}{1}
\pagenumbering{arabic}
\pagestyle{plain}
\fancyfoot[C]{\thepage}

\section{问题背景与重述}

\subsection{问题背景}

编写该模板是为了应对接下来所面对的一切中文数学建模竞赛。

众所周知,Word的公式排版奇丑无比,并且小节和公式的编号较为麻烦。
相比之下,我更喜欢\LaTeX 干净简洁的风格。
为了彻底摆脱Word,特地编写了该模板,用于接下来的各种竞赛。

\subsection{问题重述}

我们所需要解决的问题如下。
\begin{itemize}
    \item 制作出一个适用于中文建模竞赛的\LaTeX 模板;
    \item 在模板中,应当能够使用表格、图片、公式等对象。
\end{itemize}

\section{问题分析}

\subsection{问题1的分析}

在这里写问题1的分析。

\subsection{问题2的分析}

在这里写问题2的分析。

\section{模型准备}

\subsection{模型假设}

为了建立模型,我们提出如下的假设。

\begin{enumerate}
    \item 这里是第一条假设。
    \\\textbf{理由:}这里是作出第一条假设的理由。
    \item 这里是第二条假设。
    \\\textbf{理由:}这里是作出第二条假设的理由。
    \item 这里是第三条假设。
    \\\textbf{理由:}这里是作出第三条假设的理由。
    \item 这里是第四条假设。
    \\\textbf{理由:}这里是作出第四条假设的理由。
\end{enumerate}

\subsection{符号说明}

所使用的符号及说明如表\ref{table1}所示。

\begin{table}[h]
    \caption{符号说明}\label{table1}
    \centering
    \begin{tabular}{clc}
        \hline
        \textbf{符号} & \textbf{说明}        & \textbf{单位}   \\ \hline
        符号1         & 这里是符号1的说明。  & 单位             \\
        符号2         & 这里是符号2的说明。  & 单位             \\
        符号3         & 这里是符号3的说明。  & 单位             \\ \hline
    \end{tabular}
\end{table}

\section{模型的建立与求解}

\subsection{模型1}

针对问题1,建立了模型1。

其中,公式的书写方式如下。
\begin{equation}
    \label{eq1}
    {\rm{e}}^{i\theta}=\cos\theta+i\sin\theta.
\end{equation}
公式\ref{eq1}就是大名鼎鼎的Euler公式。

\subsection{模型2}

针对问题2,建立了模型2。

在论文中可能需要插入图片,在这里插入图片的方式如下。

\begin{figure}[htbp]
    \centering
    \includegraphics[width=10cm]{scientist-holding-agar-pietri-dish.jpg}
    \caption{微生物}\label{fig1}
\end{figure}

图\ref{fig1}是在实验室中,科学家拿着微生物的照片。

\section{结果的分析与检验}

\subsection{问题的结果}

在这里写问题的结果。

\subsection{模型的检验}

在这里写对模型的检验。

\section{模型的优缺点分析}

\subsection{模型的优点}

该模型具有如下的优点。
\begin{itemize}
    \item 优点1;
    \item 优点2;
    \item 优点3。
\end{itemize}

\subsection{模型的缺点与改进}

与此同时,该模型也具有如下的缺点。
\begin{itemize}
    \item 缺点1;
    \item 缺点2。
\end{itemize}
同时,在这里给出进一步优化模型的思路。

\begin{thebibliography}{99}
    \bibitem{a}作者. \emph{文献}[M]. 地点:出版社,年份.
    \bibitem{b}作者. \emph{文献}[M]. 地点:出版社,年份.
\end{thebibliography}

\begin{center}
    \Large{\textbf{附录}}
\end{center}

\begin{appendices}
    \renewcommand{\thesection}{\Alph{section}}
    \section{所用软件}
        论文使用\LaTeX 排版。
    \section{代码}
        所使用的代码如下。
\begin{lstlisting}
    Hello. 
\end{lstlisting}
\end{appendices}

\end{document}