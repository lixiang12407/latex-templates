\documentclass[UTF8,12pt]{ctexart}

\usepackage{amsmath}
\usepackage{geometry}
\geometry{left=2cm,right=2cm,top=2.5cm,bottom=2cm}

\title{Erasure Code}
\author{Lee}
\date{\today}

%---导言区---
\usepackage{fancyhdr} %调用宏包

% ---基本设置---

%设定页面的页眉页脚类型,$\LaTeX$内置了四种:empty、plain、headings及myheadings,但是我们现在不用这些内置的样式。
\pagestyle{fancy}

%清除原页眉页脚样式
\fancyhf{} 

%R:页面右边;O:奇数页;\leftmark:表示“一级标题”
\fancyhead[RO]{\leftmark}

%L:页面左边;E:偶数页;\rightmark:表示“二级标题”
\fancyhead[LE]{\rightmark}

%C:页面中间
\fancyhead[CO, CE]{Title}

%同上,但是不同位置放置不同信息
\fancyhead[LO, RE]{info}

% 设置页脚,页眉的位置上也可以放置页码
\fancyfoot[RO, LE]{80301920}
\fancyfoot[LO, RE]{info}

% 设置页眉页脚横线及样式
%页眉线宽,设为0可以去页眉线
\renewcommand{\headrulewidth}{0.3mm} 
%页脚线宽,设为0可以去页眉线
\renewcommand{\footrulewidth}{0.1mm} 

\begin{document}
RS code是基于有限域的一种编码算法,有限域又称为Galois Field,是以法国著名数学家Galois命名的,在RS code中使用$GF(2^w)$,其中$2^w \ge n + m$。 

RS code定义了一个$(n+m) * n$的Distribution Matrix。

\[
\boldsymbol{B} = \left[
\begin{matrix}
1 & 0 & 0 & 0 & 0\\
0 & 1 & 0 & 0 & 0\\
0 & 0 & 1 & 0 & 0\\
0 & 0 & 0 & 1 & 0\\
0 & 0 & 0 & 0 & 1\\
B_{11} & B_{12} & B_{13} & B_{14} & B_{15}\\
B_{21} & B_{22} & B_{23} & B_{24} & B_{25}\\
B_{31} & B_{32} & B_{33} & B_{34} & B_{35}
\end{matrix}
\right]
\]

对于每一段数据,都可以通过$\boldsymbol{B}*\boldsymbol{D}$得到。

\[
\left[
\begin{matrix}
1 & 0 & 0 & 0 & 0\\
0 & 1 & 0 & 0 & 0\\
0 & 0 & 1 & 0 & 0\\
0 & 0 & 0 & 1 & 0\\
0 & 0 & 0 & 0 & 1\\
B_{11} & B_{12} & B_{13} & B_{14} & B_{15}\\
B_{21} & B_{22} & B_{23} & B_{24} & B_{25}\\
B_{31} & B_{32} & B_{33} & B_{34} & B_{35}
\end{matrix}
\right]
\times
\left[
\begin{matrix}
D_1\\
D_2\\
D_3\\
D_4\\
D_5
\end{matrix}
\right]
=
\left[
\begin{matrix}
D_1\\
D_2\\
D_3\\
D_4\\
D_5\\
C_1\\
C_2\\
C_3
\end{matrix}
\right]
\]

假如$D_1$、$D_4$、$C_2$失效,通过从矩阵$\boldsymbol{B}$和$\boldsymbol{B}*\boldsymbol{D}$中去掉相应的行($\boldsymbol{B'}$和Survivors,可以得到如下等式:

\[
\left[
\begin{matrix}
0 & 1 & 0 & 0 & 0\\
0 & 0 & 1 & 0 & 0\\
0 & 0 & 0 & 0 & 1\\
B_{11} & B_{12} & B_{13} & B_{14} & B_{15}\\
B_{31} & B_{32} & B_{33} & B_{34} & B_{35}
\end{matrix}
\right]
\times
\left[
\begin{matrix}
D_1\\
D_2\\
D_3\\
D_4\\
D_5
\end{matrix}
\right]
=
\left[
\begin{matrix}
D_2\\
D_3\\
D_5\\
C_1\\
C_3
\end{matrix}
\right]
\]


\[
\boldsymbol{B'} \times \boldsymbol{D} = \boldsymbol{Survivors}
\]


等式左右两端乘以$\boldsymbol{B'}$矩阵的逆,即可求得$\boldsymbol{D}$

%\[
%\boldsymbol{B'^{-1}}\bodysymbol{B}\boldsymbol{D} = \boldsymbol{B'^{-1}}}\boldsymbol{Survivors}
%\]

\end{document}